\section{代码结构}
\label{sec:code-structure}
以下为压缩包中代码文件的结构说明(平台提供的文件的说明略去):
\begin{lstlisting}[language={bash}, numbers=none]
    .
    ├── bit2sym.m                 # 比特-符号映射
    ├── calculate_coding_stats.m  # 计算编码统计信息(用于Huffman编解码)
    ├── constellation_map.m       # 构建比特-符号星座图
    ├── cplxchan.m                # 复采样信道实现
    ├── dct2D.m
    ├── decode_huffman.m          # Huffman解码(未使用)
    ├── decode_vlc1.m             # 单符号VLC解码
    ├── decode_vlc2.m             # 双符号VLC解码
    ├── encode_huffman.m          # Huffman编码(用于生成码供VLC编解码)
    ├── encode_vlc1.m             # 单符号VLC编码
    ├── encode_vlc2.m             # 双符号VLC编码
    ├── idct2D.m
    ├── jpeg_dequantization.m     # JPEG反量化
    ├── jpeg_quantization.m       # JPEG量化
    ├── lena_128_bw.bmp
    ├── PSNR.m
    ├── qf_test.m                 # 联合调试主程序
    ├── qf_test_log_double.txt    # 联合调试双符号Huffman编码结果日志
    ├── qf_test_log_single.txt    # 联合调试单符号Huffman编码结果日志
    ├── seqcplxchan.m             # 复电平序列信道实现
    ├── sim_cplxchan.m            # 复电平信道仿真
    ├── sim_err_ser_vs_snr.m      # 采用信道估计后复电平序列信道误码率仿真
    ├── sim_seqcplxchan.m         # 复电平序列信道仿真
    ├── sim_snr_compare.m         # 复电平序列信道和内核的复采样信道的信噪比比较仿真      
    ├── sym2bit.m                 # 符号-比特反映射
    └── utils                     # 辅助函数目录
        └── getmax.py             # 从联合调试日志文件中获取最大PSNR
\end{lstlisting}
